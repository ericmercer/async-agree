
\newcommand{\globally}{\ensuremath{\mathbf{G}}}
\newcommand{\historically}{\ensuremath{\mathbf{H}}}
\newcommand{\assumes}{\ensuremath{A}}
\newcommand{\guarantees}{\ensuremath{P}}
\newcommand{\dispatch}{\ensuremath{\mathit{dispatch}}}
\newcommand{\complete}{\ensuremath{\mathit{complete}}}
\newcommand{\same}[1]{\ensuremath{\mathit{same}(#1)}}
\newcommand{\inputs}{\ensuremath{I}}
\newcommand{\outputs}{\ensuremath{O}}
\newcommand{\system}{\ensuremath{S}}
\newcommand{\components}{\ensuremath{C}}
\newcommand{\schedule}{\ensuremath{\phi}}
\newcommand{\valid}{\ensuremath{\mathit{valid}}}

For the reasoning framework, we use past-time linear temporal logic (PLTL) to formulate the correctness obligations.
PLTL defines operators  $\globally$ (globally) and $\historically$ (historically) over paths $\pi$ and a moment of evaluation $i$ as follows:
\begin{equation*}
 (\pi, i) \models \globally(f) \iff \forall j \ge i, (\pi, j) \models f
\end{equation*}
\begin{equation*}
(\pi, i) \models \historically(f) \iff \forall 0 \le j \le i, (\pi, j) \models f
\end{equation*}

A component is associated with a contract $(\assumes_c, \guarantees_c)$, where $\assumes_c$ are the component assumptions and $\guarantees_c$ are the component guarantees when the assumptions hold.
Contracts are written over propositions using the \emph{property specification language} (PSL).
An implementation of a component is correct if and only if $\globally(\historically(\assumes_c) \Rightarrow \guarantees_c)$.
Component implementations are assumed correct in the compositional reasoning AGREE framework.

The behavior of a scheduled component is defined by four special propositions: $\dispatch$ denoting when assumptions must hold, $\complete$ denoting when guarantees must hold, $\same{\inputs}$ and $\same{\outputs}$ denoting that the component outputs are unchanged from the previous instant. For convenience in presentation
\begin{eqnarray*}
  \dispatch(A_c) &=& \left[\left(\dispatch \wedge \assumes_c\right) \vee \left(\delta(\dispatch, \complete) \wedge \same{\inputs}\right)\right] \\
  \complete(P_c) &=& \left[\left(\complete \wedge \guarantees_c\right) \vee \left(\neg\complete \wedge \same{\outputs}\right)\right]
\end{eqnarray*}
The assumptions must hold at dispatch and the inputs may not change during activation.
%(I think this is too strong and is only needed a the system level). 
The guarantees must hold at complete and the outputs must not change between completions.
A scheduled component is correct if and only if
\[
  \globally\left(
    \historically\left(\dispatch(\assumes_c)\right) \Rightarrow \complete(\guarantees_c) \right)
\]
A component is said to be \emph{scheduled} if and only if every $\dispatch$ is eventually closed with a $\complete$ and the $\dispatch$ eventually holds at some instant.
A component is assumed correct in the compositional reasoning.

A system is defined by its own contract, the contracts from each of its constituent components that implement it, and a schedule, $\system = (\assumes_s, \guarantees_s, \components, \schedule)$ where $\components$ are the constituent component contracts and $\schedule$ defines the $\dispatch$ and $\complete$ sequences for each component.
Components \emph{communicate} in the sense that their formulas may refer to the same variables.

%A schedule is said to be \emph{fair}, $\valid(\schedule)$ if and only if every component in the system is scheduled. 

A system is correct if and only if all of the following holds: %$\valid(\schedule) \Rightarrow$
\begin{enumerate}
	 \item $ \forall (\assumes_i, \guarantees_i) \in \components, \globally(\historically(\dispatch(A_s) \wedge \bigwedge_{(\assumes_j,\guarantees_j) \in \components} \complete(P_j)) \Rightarrow \dispatch(\assumes_i)) $
	 \item $ \globally(\historically(\dispatch(A_s) \wedge \bigwedge_{(\assumes_j,\guarantees_j) \in \components} \complete(P_j)) \Rightarrow \complete(\guarantees_s))) $	
\end{enumerate}

%\[
%\begin{array}{ll}
%  & \forall (\assumes_i, \guarantees_i) \in \components, \\
%  & \ \ \ \ \globally(\historically(\dispatch(A_s) \wedge \bigwedge_{(\assumes_j,\guarantees_j) \in \components} \complete(P_j)) \Rightarrow \dispatch(\assumes_i)) \\
%  \wedge & 
%  \globally(\historically(\dispatch(A_s) \wedge \bigwedge_{(\assumes_j,\guarantees_j) \in \components} \complete(P_j)) \Rightarrow \complete(\guarantees_s)))
%\end{array}
%\]
A correct system has system assumptions and component guarantees strong enough to meet each component assumptions and the system guarantees. These ensure the correctness of the following statement for the system composed of scheduled components.
\[
  \globally\left(
    \historically\left(\dispatch(\assumes_s)\right) \Rightarrow \complete(\guarantees_s) \right)
\]