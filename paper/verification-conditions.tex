
\newcommand{\globally}{\ensuremath{\mathbf{G}}}
\newcommand{\historically}{\ensuremath{\mathbf{H}}}
\newcommand{\assumes}{\ensuremath{A}}
\newcommand{\guarantees}{\ensuremath{P}}
\newcommand{\dispatch}{\ensuremath{\mathit{dispatch}}}
\newcommand{\complete}{\ensuremath{\mathit{complete}}}
\newcommand{\same}[1]{\ensuremath{\mathit{same}(#1)}}
\newcommand{\inputs}{\ensuremath{I}}
\newcommand{\outputs}{\ensuremath{O}}
\newcommand{\system}{\ensuremath{S}}
\newcommand{\components}{\ensuremath{C}}
\newcommand{\schedule}{\ensuremath{\phi}}
\newcommand{\valid}{\ensuremath{\mathit{valid}}}
\newcommand{\dpred}{\ensuremath{\delta^\phi}}
\newcommand{\dispred}{\ensuremath{\mathbb{D}^\phi}}
\newcommand{\compred}{\ensuremath{\mathbb{C}^\phi}}

Scheduled components lend themselves to hierarchical assume-guarantee reasoning in a manner similar to that in \cite{AGREE2}.
The verification conditions to prove a system of unscheduled components correct are formalized in \emph{past-time linear temporal logic} (PLTL). 
The two PLTL operators necessary for the verification conditions are $\globally$ (globally) and $\historically$ (historically).
These are defined over a trace of the system, $\pi$, and a moment of evaluation in the trace, $i$, as follows:
\begin{eqnarray*}
 (\pi, i) \models \globally(f) & \iff & \forall j \ge i, (\pi, j) \models f \\
(\pi, i) \models \historically(f) & \iff & \forall 0 \le j \le i, (\pi, j) \models f
\end{eqnarray*}
Globally is invariant from the current moment into the future and historically is invariant from the beginning to the current moment.

We define $\mathbb{I}_c$ to be the set of components providing input to some component $c$ in the system, and we define $\mathbb{O}$ to be the set of components that provide the output for the system. An unscheduled system, $\system = (\inputs, \outputs, \assumes, \guarantees)$, is correct if and only if for all $\pi$ and for all $i \ge 0$ the following holds:
\[
\begin{array}{lll}
        & \forall c \in \components &  
            \globally(\historically(\assumes \wedge 
            \bigwedge_{c^\prime \in \mathbb{I}_c} P_{c^\prime}) 
            \implies \assumes_c) \\
 \wedge &   & 
            \globally(\historically(\assumes \wedge 
            \bigwedge_{c^\prime \in \mathbb{O}} \guarantees_{c^\prime}) 
            \implies \guarantees)
\end{array}
\]
The first condition checks the input assumptions on each component under the system assumptions and upstream component guarantees. The second checks the output guarantees of the system under the system assumptions and component guarantees providing the output.  If both conditions hold, then the system is said to be \emph{correct} meaning that $\globally(\historically(\assumes) \implies \guarantees)$ holds.

The verification conditions are extended to scheduled components by adding a notion of dispatch and complete to the verification conditions.
We define $\same{X}$ to be a predicate that is true in the very first moment and after that, it returns true at the current moment if and only if the signals in the set $X$ are unchanged from the previous moment.
We also define the predicate $\dpred_c$ to be true if the current moment is in a dispatch interval for the component $c$ according the schedule.

The assumptions in a scheduled component hold at dispatch, and the guarantees of the same component hold at complete.
A component also assumes its inputs are invariant through the dispatch interval and the outputs are invariant between complete cycles.
These requirements are captured in the following predicates where $x$ is either a system or a component:
\begin{eqnarray*}
  \dispred_x(\assumes_x) &=& \left[\left(\dispatch^\schedule_x \wedge \assumes_c\right) \vee \left(\dpred_x \wedge \same{\inputs_x}\right)\right] \\
  \compred_x(\guarantees_x) &=& \left[\left(\complete^\schedule_x \wedge \guarantees_x\right) \vee \left(\neg\complete^\schedule_x \wedge \same{\outputs_x}\right)\right]
\end{eqnarray*}
The assumptions must hold at dispatch and the inputs may not change during activation.
%(I think this is too strong and is only needed a the system level). 
The guarantees must hold at complete and the outputs must not change between completions.
These are true for components or systems.

A scheduled system, $\system = (\inputs, \outputs, \assumes, \guarantees, \schedule)$, is correct if and only if for all $\pi$ and for all $i \ge 0$ the following holds:
\[
\begin{array}{lll}
        & \forall c \in \components &  
            \globally\left[\historically\left(\dispred_S\left(\assumes\right) \wedge 
            \bigwedge_{c^\prime \in \mathbb{I}_c} \compred_{c^\prime}\left(P_{c^\prime}\right)\right) 
            \implies \dispred_c\left(\assumes_c\right)\right] \\ \\
 \bigwedge &   & 
            \globally\left[\historically\left(\dispred_S\left(\assumes\right) \wedge 
            \bigwedge_{c^\prime \in \mathbb{O}} \compred_{c^\prime}\left(\guarantees_{c^\prime}\right)\right)
            \implies \compred_{\system}\left(\guarantees\right)\right]
\end{array}
\]
Here the system itself needs a dispatch cycle in the schedule to ensure the input hold through that cycle and the output hold between cycles. As before, if both conditions hold, then a scheduled system is said to be correct meaning that $\globally\left[\historically\left(\dispred_S\left(\assumes\right)\right) 
\implies \compred_{\system}\left(\guarantees\right)\right]$ holds.
