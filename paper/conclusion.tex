{\bf Discussion.}
The model of a schedule can be simplified to potentially improve the formal verification performance. The thread execution time can be modelled as just one tick. The idle time can be abstracted out. For the same example above, a simplified schedule could be  $\sigma' = \{d(A)=(1,0,0)^*, c(A) = (0,1,0)^*, d(B) = (0,1,0)^*, c(B) = (0,0,1)^*\}$. As long as the execution order is preserved, it is straightforward to show the proof/disproof is equavilent to a model directly mapping each base time unit to a tick. It is possible to show that any counter-example found in the proof of the system property of model with schedule $\sigma$ can be mapped to a counter-example of the proof of the same system property of the same model with schedule $\sigma'$,  and vice versa. A schedule could also be specified as a set of precedence constraints, instead of an explicit sequence. However, this will potentially cause performance issues.