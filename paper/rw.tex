The AADL standard by iteself does not have a well-defined execution semantics. In order to formally verify an AADL model, it is often translated to a formal model. Then a formal method is applied to analyze the translated model.\cite{AADL2TASM}, \cite{AADL2TLA} \cite{AADS} \cite{AADL2BIP}

Model checking of concurrent processes with specific schedules or scheduling constraints is a relatively new research area. Metzler et al. \cite{Metzler2020} use an iterative and incremental approach to prove safety properties of concurrent programs. It starts with a proof under a specific schedule, and then in each following iteration gradually relaxes the scheduling constraints. The iteration stops when all possible exections are explored or a counterexample is generated. Unlike our component model, their programs are ``white boxes'', allowing their schedule to interleave instructions between programs. However, in each iteration the model checking problem is 
still challenging. In this context, our compositional verification approach makes sense.
